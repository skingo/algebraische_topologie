%19.1.2018???

\begin{lemma}
	Seien $\C, \C'$ nicht-negative Kettenkomplexe und weiter
	\begin{itemize}
		\item $C_k$ frei für $k \geq 1$
		\item $H_k(\C') = 0$ für $k \geq 1$
	\end{itemize}
	\begin{enumerate}
		\item Seien $f,g : \C \rightarrow \C'$ Kettenabbildungen mit $f_0 = g_0$, dann sind $f$ und $g$ kettenhomotop.
		\item Sei $\phi : \C_0 \rightarrow C_0'$ ein Homomorphismus mit $\phi(B_0) \subseteq B_0'$.
		Dann existiert eine Kettenabbildung $f : \C \rightarrow \C'$ mit $f_0 = \phi$.
	\end{enumerate}
\end{lemma}
\begin{proof}
	\begin{enumerate}
		\item (letzter Aufschrieb)
		\item Betrachte
		
		\begin{center}
		\begin{cd}
			\C_k \ar[r, "f_k"] \ar[d, "\del_k", swap]& \C'_k \ar[d, "\del'_k"] \\
			\C_{k-1} \ar[r, "f_{k-1}"] & \C'_{k-1}
		\end{cd}
		\end{center}
		Beweis per Induktion über $k$.
		\begin{itemize}
			\item ($k = 0$) $f_0 := \phi$ (trivial)
			\item ($k = 1$) Sei $(x_i)_{i \in I}$ eine Basis von $\C_1$ und $z_i := \phi \circ \del_1(x_i) \in B'_0 \subseteq \C'_0$ (nach Voraussetzung).
			Sei $x'_i \in \C'_1$ mit $\del'_1 x'_i = z'_i$.
			Setze dann $f_1(x_i) := x'_i$. Nach Konstruktion gilt:
			\begin{align*}
				\del'_1 \circ f_1 (x_i) &= \del'_1(x'_i) \\
				&= z'_i \\
				&= \phi \circ \del_1 (x_i) \\
				&= f_0 \circ \del_1 (x_i)
			\end{align*}
			Setze dann $f_1 : \C_1 \rightarrow \C'_1$ zu (eindeutigem) Homomorphismus fort und erhalte
			\begin{equation*}
				\del'_1 \circ f_1 = f_0 \circ \del_1
			\end{equation*}
			\item ($k \rightarrow k + 1$) Seien $f_0 \ldots f_k$ mit $\del'_j \circ f_j = f_{j-1} \circ \del_j$  für $ 1 \leq j \leq k$ schon gegeben.
			Da $\C_{k+1}$ frei ist existiert eine Basis $(x_i)_{i\in I}$ von $\C_{k+1}$.
			
			$\Rightarrow$ $z'_i := f_k \circ \del_{k+1}(x_i) \in \C'_{k}$
			
			Daraus folgt:
			\begin{align*}
				\del'_k z'_i &= \del'_k f_k \circ \del_{k+1}(x_i)  \\
				&\overset{IV}{=} f_{k-1} \circ \del_k \circ \del_{k+1} (x_i) \\
				&= 0
			\end{align*}
			
			Da $H_k(\C') = 0$ ist, existiert also ein $x'_i \in \C'_{k+1}$ mit $\del'_{k+1}(x'_i) = z'_i$.
			Setze $f_{k+1}(x_i) = x'_i$.
			
			$\Rightarrow$ (nach Konstruktion:)
			\begin{align*}
				\del'_{k+1} \circ f_{k+1}(x_i) &= \del'_{k+1} (x'_i) \\
				  &= z'_i \\
				  &= f_k \circ \del_{k+1} (x_i)
			\end{align*}
			Setze dann $f_{k+1} : \C_{k+1} \rightarrow \C'_{k+1}$ zu einem Homomorphismus fort und erhalte
			\begin{align*}
				\del' \circ f = f \circ \del
			\end{align*}
		\end{itemize}
	\end{enumerate}
\end{proof}

\begin{defn}[Eilenberg-Zilber Transformation]
	Seien $F_1,F_2 : \Top \times \Top \rightarrow \KK$ die Funktoren die auf den Objekten durch
	\begin{align*}
		F_1(X,Y) &= S(X) \tensor S(Y) \\
		F_2(X,Y) &= S(X \times Y)
	\end{align*}
	und auf den Morphismen auf naheliegende Weise gegeben sind.
	
	Man nennt eine nat. Transformation $P = (P_{XY})$ von $F_1$ nach $F_2$ eine Eilenberg-Zilber Transformation (EZ-Trafo) wenn sie \emph{normiert} ist, d.h.
	\begin{equation*}
		(P_{XY})_0(x \tensor y) = (x,y) \quad \forall x \in X \quad \forall y \in Y
	\end{equation*}
	wobei wir (wie üblich) $\Sigma_0 (x)$ mit $x$ identifizieren.
	
	\begin{center}
	\begin{cd}
		S(X_1) \tensor S(Y_1) \ar[r, "S_f \tensor S_g"] \ar[d, "P_{X_1Y_1}", swap] & S(X_2) \tensor S(Y_2) \ar[d, "P_{X_2Y_2}"] \\
		S(X_1 \times Y_1) \ar[r, "S(f\times g)", swap] & S(X_2 \times Y_2)
	\end{cd}
	\end{center}
\end{defn}

\begin{proposition}
	Seien $P, P'$ EZ-Trafos. Dann sind $P, P'$ kettenhomotop für jedes Paar $X,Y$: $P_{XY} \homot P'_{XY}$.
\end{proposition}
\begin{proof}X
	\begin{enumerate}
	\item Seien $p,q \in \N_0$ und zunächst $X = \bigtriangleup^p$ und $Y = \bigtriangleup^q$.
	Da $P$ und $P'$ normiert sind, stimmen $(P_{XY})_0$ und $(P'_{XY})_0$ überein.
	Weil $S(X) \tensor S(Y)$ frei ist und $X\times Y = \bigtriangleup^p \times \bigtriangleup^q$ ist azyklisch, denn $\bigtriangleup^p \times \bigtriangleup^q \cong \B^{p+q}$.
	
	Wegen Teil a) des Lemmas gibt es deshalb eine Kettenhomotopie
	\begin{equation*}
		D_{pq} := D_{\bigtriangleup^p \bigtriangleup^q} : S(X) \tensor S(Y) \rightarrow S(X \times Y)
	\end{equation*}
	von $P_{\bigtriangleup^p \bigtriangleup^q}$ nach $P'_{\bigtriangleup^p \bigtriangleup^q}$, also:
	\begin{equation*}
		\del D_{pq} + D_{pq}\del = P'_{pq} - P_{pq}.
	\end{equation*}
	\item Seien $X$ und $Y$ nun beliebig.
	Seien $k \in \N$, $p,q \in \N_0$ mit $p + q = k$.
	Für $\sigma \in \Sigma_p(X)$ und $\tau \in \Sigma_q(Y)$ setzen wir
	\begin{align*}
		(D_{XY})_k &: (S(X) \tensor S(Y))_k &&\rightarrow (S(X\times Y))_{k+1} \\
		&: (\sigma \tensor \tau) &&\mapsto S(\sigma \times \tau)((D_{pq})_k (id_p \tensor id_q))
	\end{align*}
	mit $id_p := id_{\bigtriangleup^p}$ und $id_q := id_{\bigtriangleup^q}$.
	Dann wird das folgende Diagramm kommutativ
	\begin{center}	
	\begin{cd}
		S(\bigtriangleup^p) \tensor S(\bigtriangleup^q) \ar[r, "D_{pq}"] \ar[d, "S\sigma \tensor S\tau", swap] & S(\bigtriangleup^p \times \bigtriangleup^q) \ar[d, "S(\sigma \times \tau)"]\\
		S(X) \tensor S(Y) \ar[r, "D_{XY}", swap] & S(X \times Y)
	\end{cd}
	\end{center}
	da gilt
	\begin{align*}
		\sigma \tensor \tau = S\sigma \tensor S\tau (id_p \tensor id_q)
	\end{align*}
	
	\underline{Behauptung:} $D_{XY}$ ist tatsächlich Kettenhomotopie zwischen $P_{XY}$ und $P'_{XY}$.
	Sei dazu $\sigma \in \Sigma_p(X)$, $\tau \in \Sigma_q(Y)$.
	Dann folgt:
	\begin{align*}
		& (\del \circ D_{XY} + D_{XY} \circ \del) (\sigma \tensor \tau) \\
		=\, & \del S(\sigma \tensor \tau) D_{pq}(id \tensor id) + D_{XY} \del (S\sigma \tensor S\tau (id \tensor id)) \\
		=\, & S(\sigma \tensor \tau) \circ \del \circ (D_{pq} (id \tensor id)) + D_{XY} \circ (S\sigma \tensor S\tau) \circ \del (id \tensor id)
	\end{align*}
	 Denn $S(\sigma \times \tau)$ und $S\sigma \tensor S\tau$ sind Kettenabbildungen.
	\end{enumerate}
	
	Daraus folgt
	\begin{align*}
		&(\del D_{XY} + D_{XY}\del)(\sigma \tensor \tau) \\
		=\, & [S(\sigma \times \tau)(-D_{pq} \circ \del + (P'_{pq} - P_{pq})) + D_{pq}\circ \del](\sigma \tensor \tau)
	\end{align*}
	
	weil $D_{pq}$ eine Kettenhomotopie ist und $(\ast)$, also
	\begin{align*}
		(\del D_{XY} + D_{XY} \del) (\sigma \tensor \tau) = S(\sigma \times \tau)(P'_{pq} - P_{pq})(id \tensor id)
	\end{align*}
	Weil $P, P'$ nat. Transformationen sind, ist daher
	\begin{align*}
		(\del D_{XY} + D_{XY} \del) (\sigma \tensor \tau) = (P'_{XY} - P_{XY})\underbrace{(S\sigma \tensor S\tau)(id \tensor id)}_{= \sigma \tensor \tau}
	\end{align*}
\end{proof}