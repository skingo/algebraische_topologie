\begin{kommentar}
  \begin{enumerate}
    \item
      Man sagt deshalb, dass $F = - \tensor G$ \emph{rechtsexakt} ist, das heißt exakte Sequenzen der Form $
      \begin{tikzcd}
      \cdot \arrow{r}{}
      & \cdot \arrow{r}{}
      & \cdot \arrow{r}{}
      & 0
      \end{tikzcd}
      $
      in wieder solche überführt.
    \item
      Er ist nicht \emph{exakt}, denn zum Beispiel ist
      \begin{equation*}
        \begin{tikzcd}
          0 \arrow{r}{}
          & \Z \arrow{r}{i}
          & \Q \arrow{r}{\pi}
          & \Q /\Z \arrow{r}{}
          & 0
        \end{tikzcd}
      \end{equation*}
      so ist die mit $\Z_2$ tensorierte Sequenz
      \begin{equation*}
        \begin{tikzcd}
          0 \arrow{r}{}
          & \underbrace{\Z \tensor \Z_2}_{=\Z_2} \arrow{r}{i \tensor \id}
          & \underbrace{\Q \tensor \Z_2}_{=0} \arrow{r}{\pi \tensor \id}
          & \Q /\Z \tensor \Z_2 \arrow{r}{}
          & 0
        \end{tikzcd}
      \end{equation*}
      nicht exakt.
    \item
      Das im Folgenden konstruierte Tensorprodukt ist ein Maß für die Abweichung der Injektivität in dem Sinne, das für eine kurze Sequenz (und abelsche Gruppe $G$)
      \begin{equation*}
        \begin{tikzcd}
          0 \arrow{r}{}
          & A \arrow{r}{\alpha}
          & B \arrow{r}{\beta}
          & C \arrow{r}{}
          & 0
        \end{tikzcd}
      \end{equation*}
      folgende Sequenz induziert wird, die exakt ist:
      \begin{equation*}
        \begin{tikzcd}
          0 \arrow{r}{}
          & \Tor(C,G) \arrow{r}{\Phi}
          & A \tensor G \arrow{r}{\alpha \tensor \id}
          & B \tensor G \arrow{r}{\beta \tensor \id}
          & C \tensor G \arrow{r}{}
          & 0
        \end{tikzcd}
      \end{equation*}
  \end{enumerate}
\end{kommentar}

\begin{defn}
  Seien $A$ und $G$ abelsche Gruppen und
  \begin{equation*}
    S(A):
    \begin{tikzcd}
      0 \arrow{r}{}
      & R \arrow{r}{i}
      & F(A) \arrow{r}{\pi}
      & A \arrow{r}{}
      & 0
    \end{tikzcd}
  \end{equation*}
  die Standard-Auflösung von $A$.
  Dann nennt man
  \begin{equation*}
    \Tor(A,G) \coloneqq \ker(j \tensor \id)
  \end{equation*}
  das \emph{Torsionsprodukt von $A$ mit $G$}.
  Es gilt dann mit sehr ähnlichem Beweis wie bei $\Ext(A,G)$:
\end{defn}

\begin{proposition}
  Seien $A$ und $A'$ abelsche Gruppen und $h \colon A \to A'$ ein Homomorphismus.
  Seien weiter
  \begin{equation*}
    \begin{tikzcd}
      S: 0 \arrow{r}{}
      & R \arrow{r}{}
      & F \arrow{r}{}
      & A \arrow{r}{}
      & 0\\
      S': 0 \arrow{r}{}
      & R' \arrow{r}{}
      & F' \arrow{r}{}
      & A' \arrow{r}{}
      & 0
    \end{tikzcd}
  \end{equation*}
  freie Auflösungen von $A$ und $A'$.
  Dann existiert eindeutig bestimmter Homomorphismus
  \begin{equation*}
    \Phi(h; S,S') \colon \ker(i \tensor \id) \to \ker(i' \tensor \id),
  \end{equation*}
  sodass für alle Fortsetzungen $(f,g)$ von $h$ gilt:
  Diagramm~\eqref{eqn:foo} kommutiert
  \begin{equation*}
    \label{eqn:bar}
    \tag{$\ast$}
    \begin{tikzcd}
      S: 0 \arrow{r}{}
      & R \arrow{r}{i}
          \arrow{d}{f}
      & F \arrow{r}{}
          \arrow{d}{g}
      & A \arrow{r}{}
          \arrow{d}{h}
      & 0\\
      S': 0 \arrow{r}{}
      & R' \arrow{r}{i'}
      & F' \arrow{r}{}
      & A' \arrow{r}{}
      & 0
    \end{tikzcd}
  \end{equation*}
  \begin{equation*}
    \label{eqn:foo}
    \tag{$\ast\ast$}
    \begin{tikzcd}
      0 \arrow{r}{}
      & \ker(i\tensor \id) \arrow{r}{\incl}
          \arrow{d}{\Phi}
      & R \tensor G\arrow{r}{i \tensor \id}
          \arrow{d}{f \tensor \id}
      & F \tensor G \arrow{r}{}
          \arrow{d}{g \tensor \id}
      & A \tensor G \arrow{r}{}
          \arrow{d}{h \tensor \id}
      & 0\\
      0 \arrow{r}{}
      & \ker(i' \tensor \id) \arrow{r}{\incl'}
      & R' \tensor G\arrow{r}{}
      & F' \tensor G \arrow{r}{}
      & A' \tensor G \arrow{r}{}
      & 0\\
    \end{tikzcd}
  \end{equation*}
\end{proposition}

\begin{kommentar}
  Die Existenz von $\Phi$ ist eigentlich nicht erstaunlich, da man natürlich
  \begin{equation*}
    \Phi \coloneqq {(\incl')}^{-1} \circ (f \tensor \id) \circ \incl
  \end{equation*}
  setzt (wenn man gecheckt hat, dass $\im(f \tensor \id \circ \incl) \subseteq  \ker(i' \tensor \id)$ liegt).
  Erstaunlich ist vielmehr, dass $\Phi$ nicht von der Auwahl der Fortsetzung $(f,g)$ von $h$ abhängt (vgl.\ Diskussion bei $\Ext$).
\end{kommentar}

\begin{kommentar}
  \begin{enumerate}[(a)]
      Die Zuordnung $(h;S,S') \mapsto \Phi(h; S, S')$ ist funktoriell in dem Sinne, dass gilt:
      \begin{enumerate}[(i)]
        \item
          $\Phi(\id,S,S) = \id$
        \item
          $\Phi(h' \circ h; S, S'') = \Phi(h'; S, S') \circ \Phi(h; S', S'')$
      \end{enumerate}
    \item
      Ist $h \colon A \to A'$ insbesondere ein Isomorphismus (z.B.\ $h = \id$), so muss für zwei freie Auflösungen $S$ und $S'$
      \begin{equation*}
        \ker(i \tensor \id) \underset{\Phi(h;S,S')}{\cong} \ker(i' \tensor \id),
      \end{equation*}
      denn $\Phi(h^{-1}; S', S) = {\Phi(h; S, S')}^{-1}$.
    \item
      Noch speziell folgt, dass für eine beliebige freie Auflösung
      \begin{equation*}
        \begin{tikzcd}
          0 \arrow{r}{}
          & R \arrow{r}{i}
          & F \arrow{r}{\pi}
          & A \arrow{r}{}
          & 0
        \end{tikzcd}
      \end{equation*}
      $\ker(i \tensor \id)$ (kanonisch) isomorph zu $\Tor(A,G)$ (vermöge $\Phi(\id, S, S(A))$) ist.
  \end{enumerate}
\end{kommentar}

\begin{beispiel}
  \begin{enumerate}
    \item
      Ist $A$ frei abelsch, so ist deshalb
      \begin{equation*}
        \Tor(A,G) = \triv
      \end{equation*}
      für alle abelschen Gruppen $G$, denn
      \begin{equation*}
        \begin{tikzcd}
          0 \arrow{r}{}
          & 0 \arrow{r}{}
          & A \arrow{r}{\id}
          & A \arrow{r}{}
          & 0
        \end{tikzcd}
      \end{equation*}
      ist freie Auflösung und $0 \tensor \id = 0$.
    \item
      Ist $A = A_1 \oplus A_2$ und $G$ beliebig, so gilt
      \begin{equation*}
        \Tor(A_1 \oplus A_2,G) \cong \Tor(A_1, G) \oplus \Tor(A_2, G),
      \end{equation*}
      denn sind
      \begin{equation*}
        \begin{tikzcd}
          S_1\colon 0 \arrow{r}{}
          & R_1 \arrow{r}{}
          & F_1 \arrow{r}{}
          & A_1 \arrow{r}{}
          & 0
        \end{tikzcd}
      \end{equation*}
      und
      \begin{equation*}
        \begin{tikzcd}
          S_2\colon 0 \arrow{r}{}
          & R_2 \arrow{r}{}
          & F_2 \arrow{r}{}
          & A_2 \arrow{r}{}
          & 0
        \end{tikzcd}
      \end{equation*}
      zwei freie Auflösungen, so auch
      \begin{equation*}
        \begin{tikzcd}
          S_1 \oplus S_2 \colon 0 \arrow{r}{}
          & R_1 \oplus R_2 \arrow{r}{j_1 \oplus j_2}
          & F_1 \oplus F_2 \arrow{r}{}
          & A_1 \oplus A_2 \arrow{r}{}
          & 0
        \end{tikzcd}
      \end{equation*}
      und daher:
      \begin{equation*}
        \Tor(A_1 \oplus A_2, G) \cong \ker(j_1 \oplus j_2,\tensor \id) \cong \ker(j_1 \tensor \id) \oplus \ker(j_2 \tensor \id) \cong \Tor(A_1, G) \oplus \Tor(A_2, G)
      \end{equation*}
    \item
      Für jedes $n \in \N$ ist
      \begin{equation*}
        \Tor(\Z_n,G) \cong \left\{ g \in G : n g = 0 \right\},
      \end{equation*}
      denn
      \begin{equation*}
        \begin{tikzcd}
          0 \arrow{r}{}
          & \Z \arrow{r}{\cdot n \eqqcolon \mult[n]}
          & \Z \arrow{r}{\pi}
          & \Z_n \arrow{r}{}
          & 0
        \end{tikzcd}
      \end{equation*}
      ist freie Auflösung und daher ist
      \begin{equation*}
        \Tor(\Z_n, G) \cong \ker(\mult[n] \tensor \id)
      \end{equation*}
      und wegen
      \begin{cd}
        0 \arrow{r}{}
        & \Tor(\Z_n, G) \arrow{r}{}
          \arrow{d}{\cong}
        & \Z \tensor G \arrow{r}{\mult[n] \tensor \id}
          \arrow{d}{\cong,(k,g) \mapsto kg}
        & \Z \tensor G \arrow{r}{}
          \arrow{d}{\cong}
        & \Z_n \tensor G \arrow{r}{}
          \arrow{d}{\cong}
        & 0\\
        0 \arrow{r}{}
        & \ker(\mult[n]) \arrow{r}{\overline {\incl}}
        & G \arrow{r}{\mult[n]}
        & G \arrow[r, "\overline \pi", "(k{,}g) \mapsto kg"']
        & \Z_n \tensor G \arrow{r}{}
        & 0
      \end{cd}
      ist tatsächlich $\Tor(\Z_n,G) = \ker(\mult[n])$.
    \item
      Ist $A$ endlich erzeugt und $G$ Körper der Charakteristik Null (also $A \cong \Tor(A) \oplus \Z^r$, mit $r = \rg(A) \in \N_0$), so ist wegen\ (a)-(c):
      \begin{equation*}
        \Tor(A,G) = \triv
      \end{equation*}
    \item
      Für alle $m,n \in \Z$ ist schließlich (Übung):
      \begin{equation*}
        \Tor(\Z_m,\Z_n) \cong \Z_{\mathrm{ggT}(m,n)}.
      \end{equation*}
  \end{enumerate}
\end{beispiel}

\begin{kommentar}
  Sei $G$ feste abelsche Gruppe.
  Dann wird $F = \Tor(-,G) \colon \Ab \to \Ab$ mit
  \begin{align*}
    F(A) & = \Tor(A,G) \\
    F(h) & = \Phi(h; S(A), S(B)) & \text{für $h \colon A \to B$ Homomorphismus}
  \end{align*}
  zu einem covarianten Funktor von \Ab\ auf sich selbst.
\end{kommentar}

\begin{defn}
  Sei $C = (C_k, \del_k)$ ein Kettenkomplex und $G$ eine abelsche Gruppe.
  Dann nennen wir
  \begin{equation*}
    C \tensor G \coloneqq (C_k \tensor G, \del_k \tensor \id_G)
  \end{equation*}
  den \emph{zugehörigen Kettenkomplex mit Koeffizienten in $G$}.
\end{defn}

\begin{kommentar}
  \begin{enumerate}
    \item
      $C \tensor G$ ist tatsächlich ein Kettenkomplex, denn
      \begin{equation*}
        (\del_k \tensor \id) \circ (\del_{k-1} \tensor \id) = \underbrace{\del_k \circ \del_{k-1}}_{= 0} \tensor \id = 0
      \end{equation*}
  \end{enumerate}
\end{kommentar}
