
\documentclass[parskip=half,a4paper]{scrartcl}

\usepackage[utf8x]{inputenc}
\usepackage[T1]{fontenc}
\usepackage{lmodern}
\usepackage[ngerman]{babel}

\usepackage{amsmath,amssymb,amsthm}
\usepackage{mathtools}

\usepackage{microtype}

\usepackage{enumerate}

\usepackage[]{hyperref}

\renewcommand{\labelenumi}{(\alph{enumi})}
\renewcommand{\labelenumii}{(\roman{enumii})}

\usepackage{tikz}
%\usetikzlibrary{cd}
\usepackage{tikz-cd}

\tikzset{%
  >=stealth
}
%\tikzcdset{%
  %arrow style=tikz,
  %diagrams={>={Straight Barb}}
  %%diagrams={>=stealth}
%}

%\listfiles
\usepackage{todonotes}

\newtheoremstyle{default}
{3pt}% pace above
{3pt}% Space below
{}% Body font
{}% Indent amount
{\bfseries}% Theorem head font
{.}% Punctuation after theorem head
{.5em}% Space after theorem head
{\thmnumber{(#2)} \thmname{#1}\thmnote{ (#3)}}% Theorem head spec (can be left empty, meaning ‘normal’)
\theoremstyle{default}
\newtheorem{prop}{Proposition}[section]
\newtheorem{lemma}[prop]{Lemma}
\newtheorem{erinnerung}[prop]{Erinnerung}
\newtheorem{kommentar}[prop]{Kommentar}
\newtheorem{korollar}[prop]{Korollar}
\newtheorem{beispiel}[prop]{Beispiel}
\newtheorem{zusatz}[prop]{Zusatz}
\newtheorem{defn}[prop]{Definition}
\newtheorem{bemerkung}[prop]{Bemerkung}
\newtheorem{proposition}[prop]{Proposition}
\newtheorem{vorbereitung}[prop]{Vorbereitung}
\newtheorem{motivation}[prop]{Motivation}
\newtheorem{satz}[prop]{Satz}
\newtheorem{theorem}[prop]{Theorem}

\title{Algebraische Topologie II ff}
\author{Ingo Skupin}
%\renewcommand*{\sectionformat}{\textrm\S \thesection\enskip}


\newcommand{\Hom}{\mathrm{Hom}}
\newcommand{\Ext}{\mathrm{Ext}}
\newcommand{\Tor}{\mathrm{Tor}}
\newcommand{\im}{\operatorname{im}}
%\newcommand{\ker}{\operatorname{ker}}
\newcommand{\coker}{\operatorname{coker}}
\newcommand{\id}{\operatorname{id}}
\newcommand{\incl}{\operatorname{incl}}
\newcommand{\rg}{\operatorname{rg}}
\newcommand{\mult}[1][]{\operatorname{mult}_{#1}}
\newcommand{\triv}{(0)}
\newcommand{\del}{\partial}
\newcommand{\tensor}{\otimes}
\newcommand{\homot}{\simeq}
\newcommand{\B}{\ensuremath{\mathbb{B}}}
\renewcommand{\S}{\ensuremath{\mathbb{S}}}
\newcommand{\Z}{\ensuremath{\mathbb{Z}}}
\newcommand{\T}{\ensuremath{\mathbb{T}}}
\newcommand{\N}{\ensuremath{\mathbb{N}}}
\newcommand{\F}{\ensuremath{\mathbb{F}}}
\newcommand{\R}{\ensuremath{\mathbb{R}}}
\newcommand{\Q}{\ensuremath{\mathbb{Q}}}
\newcommand{\CC}{\ensuremath{\mathbb{C}}}
\newcommand{\C}{\ensuremath{\mathcal{C}}}
\newcommand{\Ob}{\ensuremath{\mathit{Ob}}}
\newcommand{\Ab}{\typesetConcreteCat{Ab}}
\newcommand{\GAb}{\typesetConcreteCat{GAb}}
\newcommand{\Top}{\typesetConcreteCat{Top}}
\newcommand{\KK}{\typesetConcreteCat{KK}}
\newcommand{\CoKK}{\typesetConcreteCat{Co\kern1.5pt\text{\textbf{-}}KK}}
\newcommand{\typesetConcreteCat}[1]{\ensuremath{\mathbf{#1}}}

\newenvironment{cd*}{\shorthandoff{"}\begin{equation*}\begin{tikzcd}}{\end{tikzcd}\end{equation*}\shorthandon{"}}
\newenvironment{cd}{\shorthandoff{"}\begin{tikzcd}}{\end{tikzcd}\shorthandon{"}}


% https://tex.stackexchange.com/questions/224805/i-want-a-really-small-underbrace
\makeatletter
\def\smallunderbrace#1{\mathop{\vtop{\m@th\ialign{##\crcr
   $\hfil\displaystyle{#1}\hfil$\crcr
   \noalign{\kern3\p@\nointerlineskip}%
   \scriptsize\upbracefill\crcr\noalign{\kern3\p@}}}}\limits}
\makeatother


\begin{document}
\thispagestyle{empty}
\maketitle

\clearpage

%\renewcommand*{\thesection}{§\arabic{section}}

\setcounter{section}{8}
\section{Homologische Algebra}

%\input{homologische_algebra/2017-07-13.tex}

%\input{homologische_algebra/2017-07-18.tex}

%\input{homologische_algebra/2017-07-20.tex}

\section{Cohomologie}

\input{cohomologie/2017-07-20.tex}

\input{cohomologie/2017-07-25.tex}

%%%%%%%%%%%%%%%%%%%%%%%%%%%%%%%%%%%%
%%%%%%%% Beginn WS 17/18 %%%%%%%%%%%
%%%%%%%%%%%%%%%%%%%%%%%%%%%%%%%%%%%%

%\input{cohomologie/2017-11-03.tex}

%\input{cohomologie/2017-11-10.tex}

\input{cohomologie/2017-11-17.tex}

\input{cohomologie/2017-11-24.tex}

\input{cohomologie/2017-12-01.tex}

\input{cohomologie/2017-12-08.tex}

\input{cohomologie/2017-12-15.tex}

\input{cohomologie/2017-12-22.tex}

\input{cohomologie/2018-01-12.tex}

\begin{proposition}
	Sind $P$ und $P'$ Eilenberg-Zilber-Transformationen, so ist $P(X,Y) \cong P'(X,Y) \forall X, Y$.
\end{proposition}
\begin{proof}[Beweisidee] (a) Zunächst $X = \Delta^p, Y = \Delta^q$ ($p,q \in \mathbb{N}_0$). Baue zunächst Kettenhomotopie $D_{pq} : P_{pq} \cong P'_{pq}$, $P_{pq} := P_{\Delta^p\Delta^q}$. Funktioniert, weil $S(\Delta^p) \otimes S(\Delta^q)$ frei ist und weil $\Delta^p \times \Delta^q$ azyklisch ist.
	
	(b) $(X,Y) \Rightarrow D_{XY}$ soll auch eine natürliche Transformation sein, insbesondere muss folgendes Diagramm kommutieren:
		\begin{cd*}
		S(\Delta^p) \tensor S(\Delta^q) \ar[r, "D_{pq}"] \ar[d, "S(\sigma)\tensor S(\tau)"]
		& S(\Delta^p \times \Delta^q) \ar[d, "S(\sigma \times \tau)"]\\
		S(X) \tensor S(Y) \ar[r, "D_{XY}"]
		& S(X,Y)
	\end{cd*}

Ein $\sigma \in \Sigma_p(X), \sigma : \Delta^p \Rightarrow X$,\\
$\tau \in \Sigma_q(Y), \tau : \Delta^q \Rightarrow Y$.

$D_{XY}(\sigma \tensor \tau) := S(\sigma \times \tau) \circ D_{pq}(\id \tensor \id)$, denn $\sigma \tensor \tau = S_\sigma \tensor S_\tau(\id \tensor \id)$.

$\rightsquigarrow D_{XY} : P_{XY} \cong P'_{XY}$
\end{proof}
\begin{cd*}
	S(\Delta^p) \tensor S(\Delta^q) \ar[r, "D_{pq}"] \ar[d, swap, "S(\sigma)\tensor S(\tau)"]
	& S(\Delta^p \times \Delta^q) \ar[d, "S(\sigma \times \tau)"]\\
	S(X) \tensor S(Y) \ar[r, "D_{XY}"] \ar[d,swap, "S(f)\tensor S(g)"]\drar[phantom, "(\ast)"]
	& S(X,Y) \ar[d, "S(f \times g)"] \\
	S(X') \tensor S(Y') \ar[r, "D'_{XY}"]
	& S(X',Y') \\
\end{cd*}
\begin{align*}
f : X \rightarrow X'
g:Y \rightarrow X'
\end{align*}

\begin{kommentar}
	(a) $D = (D_{XY})$ wird selbst zu einer natürlichen Transformation in dem Sinne, dass das Diagramm $\ast$ für alle Morphismen $f : X \rightarrow X', g: Y \rightarrow X'$ kommutiert.
	
	Denn für $sigma \in \Sigma_p(X), \tau \in \Sigma_q(Y)$ gilt
	\begin{align*}
	& D_{X',Y'} \circ (S(f) \otimes S(g)) (\sigma \otimes \tau) =\\
	&= D_{X',Y'}(S(f) \otimes S(g))(S\sigma \otimes S\tau)(\id \tensor \id)\\
	&= D_{X'Y'}(S(f \circ sigma) \tensor S(g \circ \tau))(\id \tensor \id)\\
	&= S(f\sigma \times g\tau) D_{pq}(id\times \id) \textrm{(nach Definition von $D_{X'Y'}$)}\\
	&=S(f \times g) S(\sigma \times tau) D_{pq} (\id \tensor \id) \\
	&= S(f \times g) D_{XY} (\sigma \tensor \tau) \\
	&= D_{X'Y'} \circ (S(f) \tensor S(g)) = S(f \times g) \tensor D_{XY}.
	\end{align*}
	
	(b) Man vergleiche auch den Beweis aus Alg.~Top.~2, warum die baryzentrische Unterteilung $B_X : S(X) \rightarrow S(X)$ homotop zu $\id_{S(X)}$ ist. Auch dort wurde die Aussage zun-chst für $X = \Delta$ bewiesen, und dann die Natürlichkeit von $B = (B_x)$ benutzt, um eine natürliche Homotopie $D_X$ zwischen $B_X$ und $\id$ durch
	\begin{align*}
	D_X\sigma := S\Sigma circ D_{\Delta^k}(\id)
	\end{align*}
	zu definieren. (Ähnliches gilt beim Beweis des Homotopiesatzes.)
	
	Man nennt dies die \textit{Matheode der azyklischen Modelle}. Damit kann man nun auch die Existenz einer Eilenberg-Zilber-Transformation beweisen:
\end{kommentar}
	
	\begin{proposition}
		Es existiert eine Eilenberg-Zilber-Transformation\begin{align*}
		P = P_{XY} : S(X) \tensor S(Y) \rightarrow S(X \times Y).
		\end{align*}
	\end{proposition}
\begin{proof}
	a) Sei zunächst wieder $X = \Delta^p, Y = \Delta^q (p, q \in \mathbb{N}_0$. Ist $z \in (S(\Delta^p) \tensor S(\Delta^q))_0 = S_0(\Delta^p) \tensor S_0(\Delta^q)$ ein Rand, so ist z eine Summe von Elementen der Form
	\begin{align*}
	(x_2 - x_1) \tensor y_0 \textrm{und} x \tensor (y_2 - y_1)
	\end{align*}
	mit $x_1, x_2$ aus der selben Wegkomponente, $x,x_1,x_2 \in \Delta^p$ und $y, y_1, y_2 \in \Delta^q$.
	
	Die Abbildung
	\begin{align*}
	\phi : S_0(X) \tensor S_0(Y) &\rightarrow S_0(X \times Y)\\
	x \tensor y &\mapsto (x,y)
		\end{align*}
		bildet Ränder tatsächlich in Ränder ab, denn $(x_2, y) - (x_1,y)$ beziehungsweise $(x,y_2) - (x,y_1)$ sind Ränder in $S_0(X \times Y)$. Nach dem Lemma (Teil b) gibt es deshalb eine Kettenabbildung
		\begin{align*}
		P_{p,q} := P_{\Delta^p, \Delta^q} : S(\Delta^p) \tensor S(\Delta^q) \rightarrow S(\Delta^p \times \Delta^q)
		\end{align*}
		mit $P_{pq} = \phi$.
		
		(b) Nun setzen wir für beliebiges $(X,Y)$
		\begin{align*}
		\sigma &\in \Sigma_p(X)\\
		\tau &\in \Sigma_q(Y)
		\end{align*}
		wieder $P_{XY} : S(X) \tensor S(Y) \rightarrow S(X \times Y)$ fest durch
		
		\begin{cd*}
			S(\Delta^p) \tensor S(\Delta^q) \ar[r, "P_{pq}"] \ar[d, "S(\sigma)\tensor S(\tau)"]
			& S(\Delta^p \times \Delta^q) \ar[d, "S(\sigma \times \tau)"]\\
			S(X) \tensor S(Y) \ar[r, "P_{XY}"]
			& S(X,Y)
		\end{cd*}
	($\sigma = S\sigma(\id)$)
	
	Jetzt prüft man ähnlich wie in Kommentar (a) nach, dass
	\begin{itemize}
		\item $P_{XY}$ Kettenabbildung ist
		\item $P = P_{XY}$ sind natürliche Transformationen.
	\end{itemize}
$P$ ist auch normiert, denn für $x \ in X$ und $y \in Y$ ist
\begin{align*}
(P_{XY})_0(x \tensor y) = (\sigma_x \times \sigma_y) P_{pq}(\id_{\Delta^0} \tensor \id_{\Delta^0})(e_0 \tensor e_0)
\end{align*}
(mit $\sigma_x: e_0 \mapsto x, \sigma_y : e_0 \mapsto y$)
\begin{align*}
(\sigma_x \times \sigma_y)(e0, e0) = (\sigma_x(e_0), \sigma_y(e_0)) = (x,y)
\end{align*}
\end{proof}

\begin{kommentar}
	(a) Geht man von der Kategorie KK der Homotopie der Kategorie HKK über, so giebt es also nach den Propositionen genau eine natürliche Transformation $\hat{P}$ zwischen den Funktoren $\hat{F}_1$ und $\hat{F}_2 : \textrm{\underline{Top}} \times \textrm{\underline{Top}} \rightarrow \textrm{\underline{HKK}}$.
	\begin{align*}
	\hat{F}_1(X,Y) = S(X) \tensor S(Y) \\
	\hat{F}_2(X,Y) = S(X) \tensor S(Y)
	\end{align*}
	
	(b) auf s\textit{eh\/}r ähnliche Weise zeigt man nun die Existenz und Eindeutigkeit bis auf (natürliche) Kettenhomotopie einer natürlichen Transformation Q von $\hat{F}_1$ nach $\hat{F}_2$
	\begin{align*}
	Q_{X,Y} : S(X \times Y) \rightarrow S(X) \otimes S(Y),
	\end{align*}
	die \underline{normiert} ist, als
	\begin{align*}
	(Q_{XY})_0(x,y) = x \tensor y.
	\end{align*}
\end{kommentar}


\begin{satz}
  \begin{enumerate}
    \item
      Es gibt natürliche Transformation $Q = (Q_{X Y})$ von $F_2$ nach $F_1$, die \emph{normiert} ist, das heißt
      \begin{equation*}
        {(Q_{XY})}_0 \, (x,y) = x \tensor y
      \end{equation*}
    \item
      Sind $Q$ und $Q'$ wie unter (a), so gilt für alle $X$, $Y$:
      \begin{equation*}
        Q_{XY} \simeq Q_{XY}'
      \end{equation*}
  \end{enumerate}
\end{satz}
\begin{proof}
  \begin{enumerate}
    \item
      \begin{enumerate}
        \item
          Sei $k \in \N_0$, $X = \Delta^k$ und $Y = \Delta^k$.
          Es ist $S(\Delta^k \times \Delta^k)$ frei und nach der Künneth-Formel für Kettenkomplexe ist für $k > 0$
          \begin{equation*}
            H_k(S(\Delta^k) \tensor S(\Delta^k))
            = \left( H_0(\Delta^k) \tensor \underbrace{H_k(\Delta^k)}_{= 0} \oplus \underbrace{H_1(\Delta^k)}_{= 0} \tensor H_{k-1}(\Delta^k) \oplus \dotsb \oplus \underbrace{H_k(\Delta^k)}_{= 0} \tensor H_0(\Delta^k) \right)
            \oplus \left( \underbrace{\Tor(H_0(\Delta^k), H_{k-1}(\Delta^k)) \oplus \dotsm \oplus \Tor(H_{k-1}(\Delta^k), H_0(\Delta^k))}_{= 0} \right)
            = \triv
          \end{equation*}
          Also gibt es nach dem Lemma, Teil (b), eine Kettenabbildung
          \begin{align*}
            Q_k \colon S(\Delta^k \times \Delta^k) & \to S(\Delta^k) \tensor S(\Delta^k)
            \intertext{mit}
            {(Q_k)}_0 (x, y) & = x \tensor y
          \end{align*}
          Dann
          \begin{align*}
            \varphi \colon S_0(\Delta^k \times \Delta^k) & \to S_0(\Delta^k) \tensor S_0(\Delta^k)\\
            (x,y) & \mapsto x \tensor y
          \end{align*}
          bildet Ränder in Ränder ab:
          Ist $\gamma \colon [0,1] \to \Delta^k \times \Delta^k$ ein Weg von $(x_1,y_1)$ nach $(x_2, y_2)$, so ist $\gamma_1 \coloneqq \pi_1 \circ \gamma$ beziehungsweise $\gamma_2 \coloneqq \pi_2 \circ \gamma$ ein Weg von $x_1$ nach $x_2$ in $\Delta^k$ beziehungsweise $\gamma_2$ ein Weg von $y_1$ nach $y_2$ ($\pi_i \colon \Delta^k \times \Delta^k \to \Delta^k$ die Projektionen auf die Faktoren).
          Wegen
          \begin{align*}
            \varphi(\del \gamma) = \varphi( (x_2,y_2) - (x_1,y_1))
              & = x_2 \tensor y_2 - x_1 \tensor y_1 \\
              & = (x_2 - x_1) \tensor y_2 + x_1 \tensor (y_2 - y_1) \\
              & = \del \gamma_1 \tensor y_2 + x_1 \tensor \del \gamma_2\\
              & = \del(\gamma_1 \tensor y_2 + x_1 \tensor \del \gamma_2)
          \end{align*}
          folgt (a).
        \item
          Sei $k \in \N_0$, $X$ und $Y$ beliebig, sowie $\sigma \in \Sigma_k (X \times Y)$.
          Sei $d_k \in \Sigma_k(\Delta^k \times \Delta^k)$, also $d_k \colon \Delta^k \to \Delta^k \times \Delta^k$ das Diagonalsimplex,
          \begin{equation*}
            d_k(x) = (x, x).
          \end{equation*}
          Setze dann mit $\sigma_i \coloneqq \pi_i \circ \sigma \in \Sigma_k(\Delta^k)$:
          \begin{equation*}
            Q_{XY}(\sigma) \coloneqq (S \sigma_1 \tensor S \sigma_2) \circ Q_k(d).
          \end{equation*}
          \begin{cd*}
            \overset{d \in}{S_k(\Delta^k \times \Delta^k)}
              \ar[r, "Q_k"] \ar[d, "S(\sigma_1 \times \sigma_2)"]
            & S_k(\Delta^k) \tensor S_k(\Delta^k)
              \ar[d, "S \sigma_1 \tensor S \sigma_2"]\\
            \underset{\sigma \in}{S_k(X \times Y)}
              \ar[r, "Q_{XY}"]
            & {(S(X) \tensor S(Y))}_k
          \end{cd*}
          \todo[inline]{Achtung: für $f \colon X \to Y$ und $g \colon X \to Z$ ist $(f,g) \colon X \to Y \times Z$ und $f \times g \colon X \times X \to Y \times Z$}
          Beachte:
          \begin{align*}
            \sigma = (\sigma_1, \sigma_2)
            & = (S \sigma_1 (\id), S \sigma_2 (\id)) \\
            & = (S \sigma_1 \times S \sigma_2) (\underbrace{\id, \id}_{= d}) \\
            & = S(\sigma_1 \times \sigma_2) (d)
          \end{align*}
          und somit
          \begin{equation*}
            Q_{XY} \circ S (\sigma_1 \times \sigma_2) = (S \sigma_1 \tensor S \sigma_2) \circ Q_k
          \end{equation*}
          zunächst nur auf dem Element $d$, dann aber mit ähnlichem Argument für alle $\sigma \in \Sigma_k(\Delta^k \times \Delta^k)$.
          Es ist dann -- ählich wie bei $P_{XY}$ -- $Q_{XY}$ eine Kettenabbildung, die zudem auch natürlich und auch normiert ist.
          Somit folgt (a).
      \end{enumerate}
    \item
      Sind $Q$ und $Q'$ gegeben, so zeigt man wie bei $P$ zunächst mit dem Lemma, Teil (a), dass
      \begin{equation*}
        Q_k \simeq Q_k'
      \end{equation*}
      (auf dem azyklischen Modell $\Delta^k \times \Delta^k$) und dann wie bei $P$ auch
      \begin{equation*}
        Q_{XY} \simeq Q_{XY}'.
      \end{equation*}
  \end{enumerate}
\end{proof}

\begin{theorem}[Eilenberg-Zilber]
  \label{thm:ez}
  Sie $P = {(P_{XY})}$ eine EZ-Transformation und $Q = {(Q_{XY})}$ eine normierte natürliche Transformation von $F_2$ nach $F_1$.
  Dann gilt für alle $X,Y$:
  \begin{align*}
    Q_{XY} \circ P_{XY} & \simeq \id_{S(X) \tensor S(Y)} \\
    P_{XY} \circ Q_{XY} & \simeq \id_{S(X \times Y)}
  \end{align*}
\end{theorem}

\begin{kommentar}
  \begin{enumerate}
    \item
      Ist $P$ EZ-Transformation, so ist also
      \begin{equation*}
        P_{XY} \colon S(X) \tensor S(Y) \to S(X \times Y)
      \end{equation*}
      für alle Paare $(X,Y)$ eine Homothopie-Äquivalenz.
      Insbesondere induziert $P_{XY}$ dann einen Isomorphismus in der Homologie,
      \begin{equation*}
        {(P_{XY})}_{*} \colon H(S(X) \tensor S(Y)) \to H(X \times Y)
      \end{equation*}
    \item
      Während wir $P_{XY}$ nicht explizit angeben können, kann man für $Q_{XY}$ einen Repräsentanten konkret angeben (siehe Kommentar nach dem Beweis).
  \end{enumerate}
\end{kommentar}

\begin{proof}[Beweis von Theorem~\ref{thm:ez}]
  Seien $P$ und $Q$ gegeben.
  \begin{enumerate}[(i)]
    \item
      Betrachte dann $Q \circ P$ zunächst wieder auf dem Modellräumen $X = \Delta^p$, $Y = \Delta^q$.
      Da
      \begin{equation*}
        {({(Q \circ P)}_{\Delta^p \Delta^q})}_{0} (x \tensor y)
        = {(Q_{\Delta^p \Delta^q})}_0 (x, y)
        = x \tensor y
        = \id (x \tensor y)
      \end{equation*}
      exisitiert Kettenhomotopie nach dem Lemma Teil (a),
      \begin{equation*}
        D_{pq} \colon {(Q \circ P)}_{\Delta^p \Delta^q} \simeq \id_{S(\Delta^p) \tensor S(\Delta^q)}.
      \end{equation*}
      Setze dann wie gehabt $D$ auf beliebige Paare $(X,Y)$ derart fort, dass $(X,Y) \mapsto D_{XY}$ natürliche Transformation wird.
      Somit erhält man
      \begin{equation*}
        D_{XY} \colon {(P \circ Q)}_{XY} \simeq \id_{XY}
      \end{equation*}
    \item
      Ganz ähnlich sieht man, dass auch
      \begin{equation*}
        {(Q \circ P)}_{XY} \simeq \id.
      \end{equation*}
  \end{enumerate}
\end{proof}

\begin{defn}
  \begin{enumerate}
    \item
      Seien $p,q.k \in \N_0$ mit $0 \le p,q \le k$.
      Wir bezeichnen mit
      \begin{align*}
        {(\delta')}^p_k \colon \Delta^p & \to \Delta^k
        \intertext{und}
        {(\delta'')}^q_k \colon \Delta^q & \to \Delta^k
      \end{align*}
      die Einschränkungen der linearen Abbildungen $\R^{p+1} \to \R^{k+1}$ beziehungsweise $\R^{q+1} \to \R^{k+1}$, die durch
      \begin{align*}
        {(\delta')}^p_k (e_i) & = e_i & (i = 0, \dotsc, p)
        \intertext{und}
        {(\delta'')}^q_k (e_i) & = e_{k-q+i} & (i = 0, \dotsc, q)
      \end{align*}
      gegeben sind.
    \item
      Sei nun $k = p + q$ und $\sigma \colon \Delta^k \to X$ ein singuläres $k$-Simplex.
      Dann heißt
      \begin{equation*}
        \sigma'_p \coloneqq \sigma \circ {(\delta')}_k^p
      \end{equation*}
      die \emph{$p$-te Vorderseite von $\sigma$} und
      \begin{equation*}
        \sigma''_q \coloneqq \sigma \circ {(\delta'')}_k^q
      \end{equation*}
      die \emph{$q$-te Hinterseite von $\sigma$}.
      \todo[inline]{Bild}
  \end{enumerate}
\end{defn}

\begin{kommentar}
  Setzt man nun für topologische Räume $X$ und $Y$ sowie $K \in \N_0$ und $\sigma \in \Sigma_k(X \times Y)$, $\sigma = (\sigma_1, \sigma_2)$
  \begin{equation*}
    Q_{XY}(\sigma) \coloneqq \sum_{p = 0}^k {(\sigma_1')}_p \tensor {(\sigma_2'')}_{k-p},
  \end{equation*}
  so erhält man eine natürliche Transformation $(Q_{XY})$ von $F_2$ nach $F_1$, die normiert ist.
\end{kommentar}

\begin{defn}
  Seien $X$,$Y$ topologische Räume und $P = (P_{XY})$ eine EZ-Transformation,
  \begin{equation*}
    P_{XY} \colon S(X) \tensor S(Y) \to S(X \times Y).
  \end{equation*}
  Wir definieren das \emph{Homologie-Kreuzprodukt}
  \begin{align*}
    H(X) \times H(Y) & \to H(X \times Y)
    \intertext{durch}
    ([z], [z']) & \mapsto \left[P_{XY}(z \tensor z')\right]
  \end{align*}
\end{defn}

\begin{kommentar}
  \begin{enumerate}
    \item
      Für zwei graduierte abelsche Gruppen $G = (G_p)$, $G' = (G'_q)$ definiert man ihr Produkt
      $G \times G' = {(G \times G')}_k$ als die folgende graduierte abelsche Gruppe:
      \begin{equation*}
        {(G \times G')}_k \coloneqq \bigoplus_{p+q = k} G_p \times G'_q.
      \end{equation*}
    \item
      Das Homologie-Kreuzprodukt hängt nicht von der Auswahl der EZ-Transformation ab, denn es gilt mit der natürlichen Transformation $\lambda = (\lambda_{XY})$,
      \begin{align*}
        \lambda_{XY} \colon H(X) \tensor H(Y) & \to H(S(X) \tensor S(Y))\\
        [z] \tensor [z'] & \mapsto [ z \tensor z' ]
      \end{align*}
      für $\alpha = [z] \in H_p(X)$ und $\beta = [z'] \in H_q(Y)$:
      \begin{align*}
        \alpha \times \beta
        & = [P(z \tensor z')] \\
        & = P_* ( [z \tensor z']) \\
        & = P_* \circ \lambda_{XY} (\alpha \tensor \beta),
      \end{align*}
      und für jede weitere Wahl einer EZ-Transformation $P'$ ist $P_*' = P_*$.
  \end{enumerate}
\end{kommentar}



\begin{satz}
  Für das Homologie-Kreuzprodukt $\times = (\times_{X Y})$ gilt:
  \begin{enumerate}
    \item
      $\times$ ist natürliche Transformation zwischen den Funktoren $F_1, F_2 \colon \Top^2 \to \GAb$,
      \begin{align*}
        F_1(X,Y) & = H(X) \tensor H(Y) \\
        F_2(X,Y) & = H(X \times Y)
      \end{align*}
      (und auf Morphismen naheliegend).
    \item
      $\times_{XY} \colon H(X) \times H(Y) \to H(X \times Y)$ ist bilinear für alle $X$ und $Y$.
    \item
      Das Kreuzprodukt ist im folgenden Sinne kommutativ:
      Ist $\alpha \in H_p(X)$, $\beta \in H_q(Y)$ für $p,q \in \N_0$, so gilt mit $t_{XY} \colon X \times Y \to Y \times X, (x,y) \mapsto (y,x)$:
      \begin{equation*}
        \beta \times \alpha
         = {(-1)}^{pq} t_* (\alpha \times \beta)
      \end{equation*}
    \item
      Das Kreuzprodukt ist im folgenden Sinne assoziativ:
      Ist $\alpha \in H(X)$, $\beta \in H(Y)$ und $\gamma \in H(Z)$, so gilt in $H(X \times Y \times Z)$:
      \begin{equation*}
        (\alpha \times \beta) \times \gamma = \alpha \times (\beta \times \gamma).
      \end{equation*}
    \item
      Sei $x_0 \in X$ und $j = j_{x_0} \colon Y \to X \times Y, j(y) = (x_0,y)$.
      Dann gilt für alle $\beta \in H(Y)$:
      \begin{equation*}
        [x_0] \times \beta = j_* (\beta)
      \end{equation*}
      (Neutrales Element).
  \end{enumerate}
\end{satz}

\begin{proof}
  \begin{enumerate}
    \item
      Die Darstellung
      \begin{equation*}
        \alpha \times \beta = P_* \circ \lambda(\alpha \tensor \beta)
      \end{equation*}
      zeigt, dass $\times$ natürliche Transformation ist (Verkettung natürlicher Transformationen), das heißt:
      Für $f \colon X \to X'$ und $g \colon Y \to Y'$ stetig ist kommutativ:
      \begin{cd*}
        H(X) \times H(Y)
        \ar[r, "\times_{X Y}"] \ar[d, "f_* \times g_*"]
        & H(X \times Y)
        \ar[d, "{(f \times g)}_*"]
        \\
        H(X') \times H(Y')
        \ar[r, "\times_{X' Y'}"]
        & H(X' \times Y')\\
      \end{cd*}
    \item
      Die Formel zeigt auch, dass $\times_{XY}$ bilinear ist,
      \begin{align*}
        (\alpha_1 + \alpha_2) \times \beta
        & = \alpha_1 \times \beta + \alpha_2 \times \beta \\
        \alpha \times (\beta_1 + \beta_2)
        & = \alpha \times \beta_1 + \alpha \times \beta_2
      \end{align*}
      für alle $\alpha,\alpha_1,\alpha_2 \in H(X)$, $\beta, \beta_1, \beta_2 \in H(Y)$, denn $\tensor$ ist bilinear und $\lambda$ sowie $P$ sind linear.
    \item
      Betrachte die Kettenabbildung
      \begin{equation*}
        \tau_{XY} \colon S(X) \tensor S(Y) \to S(Y) \tensor S(X)
      \end{equation*}
      gegeben durch
      \begin{equation*}
        \tau(c \tensor d) = {(-1)}^{pq} d \tensor c
      \end{equation*}
      für $c \in S_p(X)$, $d \in S_q(Y)$.
      Wir behaupten nun, dass das folgende Diagramm bis auf Kettenhomotopie kommutiert:
      \begin{cd*}
        \label{dia:kommutativ}
        \tag{$\ast$}
        S(X) \tensor S(Y)
        \ar[r, "\tau_{XY}"]
        \ar[d, "P_{XY}"]
        & S(Y) \tensor S(X)
        \ar[d, "P_{YX}"]
        \\
        S(X \times Y)
        \ar[r, "S t_{XY}"]
        & S(Y \times X)
      \end{cd*}
      wobei wie eben $t_{XY} \colon X \times Y \to Y \times X$ durch $t(x,y) = (y,x)$ gegeben ist.
      Es folgt dann für $\alpha = [z] \in H_p(X)$ und $\beta = [z'] \in H_q(Y)$:
      \begin{align*}
        \beta \times \alpha
        & = [ P_{YX} (z' \tensor z) ] \\
        & = {(-1)}^{pq} [ P_{YX} \circ \tau_{XY} (z \tensor z') ] \\
        & \overset{\eqref{dia:kommutativ}}{=}
          {(-1)}^{pq} [ S t_{XY} \circ P_{XY} (z \tensor z') ] \\
        & = {(-1)}^{pq} {(t_{XY})}_* \circ [ P_{XY} (z \tensor z') ] \\
        & = {(-1)}^{pq} {(t_{XY})}_{*} (\alpha \times \beta)
      \end{align*}

      Zur $H$-Kommutativität von~\eqref{dia:kommutativ} benutzen wir die Methode der azyklischen Modelle:
      \begin{enumerate}
        \item
          Seien $X = \Delta^p$, $Y = \Delta^q$.
          Dann ist $S(X) \tensor S(Y)$ frei und $S(Y \times X$ azyklisch und außerdem
          \begin{align*}
            {(P_{YX} \circ \tau_{XY})}_0 (x \tensor y)
            & = {(P_{YX})}_0 (y \tensor x) \\
            & = (y,x) \\
            & = S t_{XY} (x, y) \\
            & = S t_{XY} \circ {(P_{XY})}_0 (x \tensor y)
          \end{align*}
          für alle $x \in X$, $y \in Y$.
          Somit folgt (La.)
          \begin{equation*}
            P_{YX} \circ \tau_{XY} \simeq S t_{XY} \circ P_{XY}.
          \end{equation*}

          Nun sind neben $P = (P_{XY})$ und $t = (t_{XY})$ und $\tau = (\tau_{XY})$ natürliche Transformationen und daher auch $(P_{YX} \circ \tau_{XY})$ und $(S t_{XY} \circ P_{XY})$ und daher kommutiert dann~\eqref{dia:kommutativ} bis auf Homotopie auch für alle Paare $(X,Y)$.
      \end{enumerate}
    \item
      Betrachte das Diagramm
      \begin{cd*}
        \label{dia:assoc}
        \tag{$\ast\ast$}
        S(X) \tensor S(Y) \tensor S(Z)
        \ar[r, "\id \tensor P_{YZ}"]
        \ar[d, "P_{XY} \tensor \id"]
        & S(X) \tensor S(Y \times Z)
        \ar[d, "P_{X, Y \times Z}"]
        \\
        S(X \times Y) \tensor S(Z)
        \ar[r, "P_{X \times Y, Z}"]
        & S(X \times Y \times Z)
      \end{cd*}
      Dass dieses bis auf Homotopie kommutiert, zeigt man wieder mit dem Methode der azyklischen Modelle ($X = \Delta^p$, $Y = \Delta^q$, $Z = \Delta^r$).
      Es folgt dann für $\alpha = [z] \in H(X)$, $\beta = [z'] \in H(Y)$ und $\gamma = [z''] \in H(Z)$:
      \begin{align*}
        (\alpha \times \beta) \times \gamma
        & = [ P_{XY} (z \tensor z') ] \times \gamma \\
        & = [ P_{X \times Y, Z} (P_{XY} (z \tensor z') \tensor z'') ] \\
        & = [P_{X \times Y, Z}  \circ (P_{XY} \tensor \id) (z \tensor z' \tensor z'') ] \\
        & \overset{\eqref{dia:assoc}}{=}
        [P_{X, Y \times Z} \circ (\id \tensor P_{YZ}) (z \tensor z' \tensor z'') ] \\
        & = [P_{X, Y \times Z} (z \tensor P_{YZ} (z' \tensor z'')) ] \\
        & = \alpha \times [ P_{YZ} (z' \tensor z'') ] \\
        & = \alpha \times (\beta \times \gamma)
      \end{align*}
    \item
      Methode der azyklischen Modelle:
      Sei zunächst $X = \left\{ x_0 \right\}$ und $\psi \colon S(Y) \to S(X) \tensor S(Y)$ die Kettenabbildung mit
      \begin{equation*}
        \psi(d) = x_0 \tensor d.
      \end{equation*}
      Dann kommutiert bis auf Homotopie (Beweis wieder mit dem Modellfall $Y = \Delta^q$):
      \begin{cd*}
        \label{dia:foo}
        \tag{$\ast\ast\ast$}
        S(Y)
        \ar[r, "S j"]
        \ar[d, "\psi"]
        & S(X \times Y) \\
        S(X) \tensor S(Y)
        \ar[ur, "P_{XY}"]
      \end{cd*}
      Es folgt für $\beta = [z'] \in H(Y)$:
      \begin{align*}
        [x_0] \times \beta
        & = [ P_{XY}(x_0 \tensor z') ] \\
        & = [ P_{XY} \circ \psi(z') ] \\
        & \overset{\eqref{dia:foo}}{=}
          [S j (z') ] \\
        & = j_* ( [z'] ) \\
        & = j_* ( \beta ).
      \end{align*}
      Für den allgemeinen Fall benutzt man Teil (a) mit den Abbildungen $i \colon \left\{ x_0 \right\} \xhookrightarrow{} X$, $j_0 \colon Y \to \left\{ x_0 \right\} \times Y$ und $j \colon Y \to X \times Y$.
      Wegen $(i \times \id) \circ j_0 = j$,
      \begin{cd*}
        Y
        \ar[r, "j_0"]
        \ar[d, "j"]
        & \left\{ x_0 \right\} \times Y
        \ar[dl, "i \times \id"]
        \\
        X \times Y
      \end{cd*}
      ist dann.
      \begin{align*}
        {[x_0]}_X \times \beta
        & = i_* ([x_0]) \times \beta \\
        & = (i_* \times \id_*) ([x_0], \beta) \\
        & \overset{\times \text{nat. Trans.}}{=} {(i \times \id)}_* ( [x_0] \times \beta) \\
        & = {(i \times \id)}_* ( {(j_0)}_* (\beta)) \\
        & = {( (i \times \id) \circ j_0)}_* ([\beta]) \\
        & = j_* ([\beta])
      \end{align*}
  \end{enumerate}
\end{proof}

\begin{satz}[Künneth-Formel für topologische Räume]
  Sei $\times = (\times_{XY})$ das Homologie-Kreuzprodukt, $Q = (Q_{XY})$ eine (inverse) EZ-Transformation und $\mu = (\mu_{XY} \colon H(S(X) \tensor S(Y)) \to \Tor(H(X),H(Y)))$ aus der Künneth-Formel für Kettenkomplexe.
  Dann ist die folgende Sequenz natürlich, exakt und spaltet:
  \begin{cd*}
    \label{seq:kuenneth_top}
    \tag{$\star$}
    0 \ar[r]
    & H(X) \tensor H(Y) \ar[r, "\times"]
    & H(X \times Y) \ar[r, "\mu \circ Q_*"]
    & \Tor(H(X),H(Y)) \ar[r]
    & 0.
  \end{cd*}
\end{satz}

\begin{proof}
  Sei $P = (P_{XY})$ eine EZ-Transformation.
  Dann ist $P_*$ invers zu $Q_*$.
  Benutze nun die Natürlichkeit, Exaktheit und Spaltung der folgenden Sequenz nach de Künneth-Formel für Kettenkomplexe:
  \begin{cd*}
    \label{seq:kuenneth_kk}
    \tag{$\star\star$}
    0 \ar[r]
    & H(X) \tensor H(Y) \ar[r, "\lambda_{XY}"]
    \ar[dr, "\times_{XY}"]
    & H(S(X) \tensor S(Y)) \ar[r, "\mu_{XY}"]
    \ar[d, "P_*", swap, "\cong", bend left]
    & \Tor(H(X),H(Y)) \ar[r]
    & 0 \\
    & & H(X \times Y)
    \ar[u, "Q_*", bend left]
    \ar[ur, "\mu_{XY} \circ {(Q_{XY})}_*"]
  \end{cd*}
  Die Aussagen übertragen sich daher auf die Künneth-Sequenz für topologische Räume.
\end{proof}

\begin{beispiel}
  \begin{enumerate}
    \item
      Ist insbesondere $H(X)$ \emph{oder} $H(Y)$ frei, so liefert das Homologie-Kreuzprodukt also einen Isomorphismus.
      \begin{enumerate}
        \item
          Zum Beispiel ist dann für $m,n \in \N$ mit $m \neq n$:
          \begin{equation*}
            H_k(S^m \times S^n) =
            \begin{cases}
              \Z & \text{für $k \in \left\{ 0,m,n,m+n \right\}$} \\
              0 & \text{sonst}
            \end{cases}
          \end{equation*}
          und für $m = n$:
          \begin{equation*}
            H_k(S^n \times S^n) = 
            \begin{cases}
              \Z & \text{für $k = 0$ und $k = 2n$} \\
              \Z^2 & \text{für $k = n$} \\
              0 & \text{sonst}
            \end{cases}
          \end{equation*}
        \item
          Für den $n$-dimensionalen Torus $\T = {(\S^1)}^n$ ist
          \begin{equation*}
            H_k(\T^n) =
            \begin{cases}
              \Z^{\binom{n}{k}} & \text{für $0 \le k \le n$} \\
              0 & \text{sonst}
            \end{cases}
          \end{equation*}
      \end{enumerate}
    \item
      Für $k = 1$ gilt für beliebige topologische Räume:
      \begin{equation*}
        H_1(X \times Y) = (H_0(X) \tensor H_1(Y)) \oplus (H_1(X) \tensor H_0(Y))
      \end{equation*}
      also für $X,Y$ wegzusammenhängend:
      \begin{equation*}
        H_1(X \times Y) = H_1(X) \oplus H_1(Y)
      \end{equation*}
  \end{enumerate}
\end{beispiel}


\end{document}
